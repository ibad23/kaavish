% \section{Problem Statement}

% This section explains the problem we are solving and our motivation to do so.

% \section{Proposed Solution}

% This section gives a summary of our proposed solution, i.e. how does it solve the problem? An overview of the system is provided. A detailed description of each module of the system is presented later in Chapter ~\ref{chap:intro}.

% \section{Intended User}

% This section outlines the target users of this system. The different types of users in our user base and their interaction with the system are described briefly.

% \section{Project gantt chart and deliverables}
% Please include a detailed gantt and details of the deliverables for Kaavish I and II

% \section{Key Challenges}

% This section mentions the key challenges that we foresee in this project and possible ways to address them.

One of the biggest challenges in product development today is the disconnect between design and implementation. Designers create visually appealing and user-friendly interfaces in Figma, the design tool of choice for UI/UX professionals, but translating these designs into production-ready code requires significant time, effort, and communication between design and engineering teams. Design systems are the backbone of scalable SaaS products, bringing together reusable components, design guidelines, and patterns that speed up development and enhance the user experience. Existing B2B SaaS platforms struggle to maintain a unified design system and visual consistency that would allow them to scale rapidly while integrating intelligent automation. 
\\\\
This project, Design-to-Code with Knoccs by SpurSol, seeks to solve these challenges by creating a scalable design-to-code pipeline that allows design components to automatically translate into production-ready code. This project specifically aims to revamp the Knoccs design system and develop a multi-agent AI layer capable of autonomously generating various components of the codebase, including the front-end, back-end, and database. In addition, we aim to leverage this pipeline to implement new functionalities and AI use cases within Knoccs itself, thereby validating the system's scalability, performance, and real-world applicability.
